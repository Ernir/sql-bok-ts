Fyrr í bókinni höfum við farið yfir fjórar grunnaðgerðir sem nauðsynlegar eru til gagnagrunnsvinnslu. Við kunnum að
\begin{itemize}
 \item Búa til gagnagrunna og töflur (\verb|CREATE| skipunin, kaflar \ref{undirkafli:synidaemi-i-sql} og \ref{undirkafli:bua-til-toflu})
 \item Setja gögn inn í töflur (\verb|INSERT| skipunin, \ref{undirkafli:innsetning})
 \item Ná í gögn úr töflum (\verb|SELECT| skipunin, kaflar \ref{kafli:select} og \ref{kafli:gagnavinnslamargartoflur} eins og þeir leggja sig)
 \item Eyða töflum og öllu sem í þeim er (\verb|DROP| skipunin, kafli \ref{undirkafli:drop})
\end{itemize}
Þetta kemur okkur býsna langt. Þetta hefur hins vegar ekki leyft okkur að gera nokkrar breytingar á töflum eða þeim gögnum sem í þeim eru. Til þess þurfum við fleiri skipanir. Þær heita \verb|ALTER|, \verb|UPDATE| og \verb|DELETE|.
\section{DDL og DML}
\section{Að breyta töflum}
\section{Að breyta gögnum}
\section{Að eyða gögnum}
\section{Hreyfingar} %Transactions