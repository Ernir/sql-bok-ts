Í kafla \ref{undirkafli:synidaemi-i-sql} sáum við dæmi um hvernig búa má til töflu með SQL-skipun. Hins vegar eyddum við ekki sérstaklega miklum tíma í að reyna að skilja hvernig skipunin er uppbyggð, hvað öll lykilorðin sem fram komu þýddu eða hvað er leyfilegt. Viðfangsefni kaflans sem við erum stödd í núna verður að leiðrétta þennan trassaskap og sökkva okkur í töflugerð.
\section{Að búa til töflu} % CREATE
Tökum skref aftur á bak frá og athugum hversu einfalda töflu við getum búið til. Hún er líklega eitthvað á þá leið sem sjá má á sýnidæmi \ref{sql:k3d1-create-table-fallegt}.

\begin{example}[h]
\caption{Mjög einföld tafla}
\label{sql:k3d1-create-table-fallegt}
\centering
\sql{sql/k3d1-create-table-fallegt.sql}
\end{example}

Skoðum þessa skipun nú mjög vandlega.
\begin{itemize}
 \item Hún hefst á að lýsa yfir hvað gera skal - hér er það \verb|CREATE TABLE|. Næst kemur nafn töflunnar fram.
 \item Þegar nafn töflunnar hefur verið gefið opnast svigi.
 \item Inni í sviganum kemur nafnið á einum dálki og orðið \verb|INTEGER|\footnote{Merkingin á þessum lykilorðum sem við höfum séð á eftir dálkheitunum, \emph{INTEGER} og \emph{VARCHAR}, útskýrist í næsta undirkafla (\ref{undirkafli:gagnagerdir}).}. Hér er einungis einn dálkur.
 \item Skipuninni lýkur á því að sviganum er lokað og semíkomma (;) sett á eftir.
\end{itemize}

Skoðum næst sýnidæmi \ref{sql:k3d2-create-table-meira-fallegt}, sem er \emph{örlítið} stærra.
\begin{example}[h]
\caption{Einföld tafla}
\label{sql:k3d2-create-table-meira-fallegt}
\centering
\sql{sql/k3d2-create-table-meira-fallegt.sql}
\end{example}

Hér sjáum við ágætlega hvernig bæta má við öðrum dálki. Fyrsta dálklýsingin er skrifuð, síðan kemur komma, síðan fylgir næsta dálklýsing.

Athugum að engin komma er á eftir síðustu dálklýsingunni. Það er vegna þess að dálkarnir eru einungis \emph{aðskildir} með kommum, komman er ekki hluti af dálklýsingunni sjálfri.
\section{Helstu gagnagerðir} % INTEGER (INT), VARCHAR, CHAR, DATE
\label{undirkafli:gagnagerdir}
Einni stórri spurningu um síðustu tvö dæmi er enn ósvarað - hvað er þetta \verb|INTEGER|?
\section{Tóm gildi} % NULL, NOT NULL
\section{Innsetning gagna} % INSERT
\section{Aðallyklar} %PRIMARY KEY
\section{Að eyða töflum} % DROP