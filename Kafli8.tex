\section{Views}
\section{Yfirlit yfir helstu gagnagrunnskerfi}
\label{undirkafli:helstu-gagnagrunnskerfi}
\subsection{MySQL}
\subsection{SQLite}
\subsection{PostgreSQL}
\subsection{Microsoft SQL Server}
\subsection{Oracle Database}
\section{Venslalíkanið}
\section{Að tengjast gagnagrunni með PHP}
\label{undirkafli:php}
Við höfum eytt miklum tíma í að skoða gagnagrunna sem sjálfstætt fyrirbrigði.

Í þessum undirkafla skoðum við loksins hvernig tengja má MySQL-gagnagrunn við PHP-forritskóða.
Slíkar tengingar eru teknar fyrir vegna þess hve algengar\footnote{Linux, Apache, PHP og MySQL mynda saman ``pakka'' sem oft er notaður sem ein heild. Pakkinn er nefndur eftir skammstöfun sinni, LAMP. Hann er í gríðarmikilli notkun.} þær eru í vefforritun.

Útskýringarnar gera ráð fyrir skilningi á ýmsum atriðum:
\begin{itemize}
 \item Hugtökum í vefsíðum - HTML, Javascript
 \item Keyrslu PHP-skripta
 \item Grundvallarmálfræði PHP
 \item IP-tölum
 \item Föllum, lykkjum og fylkjum\footnote{e. \emph{functions}, \emph{loops} og \emph{arrays}}
\end{itemize}
\subsection{Hlutverk gagnagrunna í vefsíðum}
Vefsíða sem byggð er upp á hefðbundinn hátt skiptist gróflega í tvo hluta - client og server. 

\begin{figure}
\centering
\caption{Hefðbundin uppbygging vefsíðu}
\label{mynd:uppbyggingvefsidu}
\color{red} Hingað kemur falleg mynd af client-server strúktúr vefsíðu.
\end{figure}

Client-hlutinn er sá hluti sem keyrður er á tölvu notandans. Í client-hluta eru HTML-tög túlkuð og Javascript-kóði keyrður, m.a.. Venjulega fer þessi vinna fram í vafra\footnote{e. \emph{browser}, t.d. Google Chrome, Firefox og Internet Explorer} notandans.

Server-hlutinn er margskiptur. Viðfangsefni okkar, PHP og MySQL, tilheyra þessum hluta. Oft keyra PHP og MySQL á sömu tölvu, sem er þá einfaldlega nefnd ``serverinn''.

Hlutverk gagnagrunnsins í þessari uppbyggingu er, eðli hans samkvæmt, það að halda utan um upplýsingar. PHP-hluti serversins sér um að eiga samskipti við gagnagrunninn og miðla upplýsingunum áfram til clientsins. Notandinn og tölva hans eiga aldrei bein samskipti við MySQL-serverinn.

\subsection{Uppsetning tengingar með PDO}
Gerum ráð fyrir að við séum með PHP-skriptu sem keyrir á vefþjóni. Til að tengjast MySQL-gagnagrunni þarf hún eftirfarandi upplýsingar:
\begin{itemize}
 \item Tengingarupplýsingar: Gagnagrunnsgerðina (hjá okkur alltaf MySQL), nafn gagnagrunnsins og staðsetningu hans
 \item Notandanafn MySQL-þjónsins
 \item Lykilorð notandans.
\end{itemize}
Klasann \href{http://www.php.net/manual/en/class.pdo.php}{PDO}\footnote{\url{http://www.php.net/manual/en/class.pdo.php}} má svo nota til þess að mynda tenginguna sjálfa.\footnote{Fleiri leiðir eru færar til að mynda tengingar af þessum toga, til dæmis \emph{mysqli} viðbótin. Þá PHP-viðbót sem einfaldlega heitir \emph{mysql} skal þó alls ekki nota - hún er úreld.}

\begin{example}[h]
\caption{Tenging við gagnagrunn með PDO}
\label{sql:k8d1}
\centering
\inputminted[frame=lines, fontfamily=courier]{php}{php/k8d1.php}
\end{example}

Dæmi um hvernig öll skriptan gæti litið út má sjá á sýnidæmi \ref{sql:k8d1}. Skoðum það sýnidæmi nánar. 

Fyrsta breytan sem er skilgreind í skriptunni (\verb|$source|) inniheldur tengingarupplýsingarnar. Þar kemur fyrst fram orðið \verb|mysql|, aðskilið frá gagnagrunnsnafninu og staðsetningu gagnagrunnins með tvípunkti. Sé ætlunin að skrifa skriptu af þessum toga til nota í eigin forriti þyrfti að skipta út \verb|testdb| fyrir nafn gagnagrunnsins sem nota skal. Sé MySQL-serverinn ekki að keyra á sömu tölvu og PHP-skriptan þarf einnig að breyta IP-tölunni í IP-tölu tölvunnar sem MySQL-serverinn keyrir á.\footnote{Server Upplýsingatækniskólans keyrir á IP-tölunni \emph{82.148.66.15}. Einnig má skrifa nafn hans, \emph{tsuts.tskoli.is}, í stað IP-tölunnar.}

Næstu tvær breytur innihalda notandanafn og lykilorð fyrir MySQL-serverinn. Þetta er sama notandanafn og lykilorð og notað var til að tengjast gagnagrunninum í kafla \ref{kafli:fyrstuskrefin}.

Þegar upplýsingarnar hafa verið geymdar er loks ``reynt'' að búa til tengingu með PDO. Takist það eru tengingarupplýsingarnar geymdar í  ``database handle'' breytu, nefnd \verb|$dbh| í sýnidæminu. Þessa breytu má síðan nýta til að eiga samskipti við gagnagrunninn.

Takist ekki að koma tengingunni á prentar skriptan út strenginn ``Tenging mistókst'' ásamt villuskilaboðunum sem berast.
\subsection{Að sækja gögn með PDO}
\subsection{Að birta gögn á vefsíðu}