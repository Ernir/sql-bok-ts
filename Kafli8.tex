\section{Views}
\section{Yfirlit yfir helstu gagnagrunnskerfi}
\subsection{MySQL}
\subsection{SQLite}
\subsection{PostgreSQL}
\subsection{Microsoft SQL Server}
\subsection{Oracle Database}
\section{Venslalíkanið}
\section{Að tengjast gagnagrunni með PHP}
Við höfum eytt miklum tíma í að skoða gagnagrunna sem sjálfstætt fyrirbrigði.

Í þessum undirkafla skoðum við loksins hvernig tengja má MySQL-gagnagrunn við PHP-forritskóða.
Slíkar tengingar eru teknar fyrir vegna þess hve algengar\footnote{Linux, Apache, PHP og MySQL mynda saman ``pakka'' sem oft er notaður sem ein heild. Pakkinn er nefndur eftir skammstöfun sinni, LAMP. Hann er í gríðarmikilli notkun.} þær eru í vefforritun.

Útskýringarnar gera ráð fyrir skilningi á ýmsum atriðum:
\begin{itemize}
 \item Hugtökum í vefsíðum - HTML, Javascript
 \item Keyrslu PHP-skripta
 \item Grundvallarmálfræði PHP
 \item Föllum, lykkjum og fylkjum\footnote{e. \emph{functions}, \emph{loops} og \emph{arrays}}
\end{itemize}
\subsection{Hlutverk gagnagrunna í vefsíðum}
Vefsíða sem byggð er upp á hefðbundinn hátt skiptist gróflega í tvo hluta - client og server. 

\begin{figure}
\centering
\caption{Hefðbundin uppbygging vefsíðu}
\label{mynd:uppbyggingvefsidu}
\color{red} Hingað kemur falleg mynd af client-server strúktúr vefsíðu.
\end{figure}

Client-hlutinn er sá hluti sem keyrður er á tölvu notandans. Í client-hluta eru HTML-tög túlkuð og Javascript-kóði keyrður, m.a.. Venjulega fer þessi vinna fram í vafra\footnote{e. \emph{browser}, t.d. Google Chrome, Firefox og Internet Explorer} notandans.

Server-hlutinn er margskiptur. Viðfangsefni okkar, PHP og MySQL, tilheyra þessum hluta. Oft keyra PHP og MySQL á sömu tölvu, sem er þá einfaldlega nefnd ``serverinn''.

Hlutverk gagnagrunnsins í þessari uppbyggingu er, eðli hans samkvæmt, það að halda utan um upplýsingar. PHP-hluti serversins sér um að eiga samskipti við gagnagrunninn og miðla upplýsingunum áfram til clientsins. Notandinn og tölva hans eiga aldrei bein samskipti við MySQL-serverinn.

\subsection{Uppsetning tengingar með PDO}
Gerum ráð fyrir að við séum með PHP-skriptu sem keyrir á vefþjóni. Til að tengjast MySQL-gagnagrunni þarf hún eftirfarandi upplýsingar:
\begin{itemize}
 \item Tengingarupplýsingar: Gagnagrunnsgerðina (hjá okkur alltaf MySQL), nafn gagnagrunnsins og staðsetningu hans
 \item Notendanafn MySQL-þjónsins
 \item Lykilorð notandans.
\end{itemize}
Klasann \href{http://www.php.net/manual/en/class.pdo.php}{PDO}\marginnote{\url{http://www.php.net/manual/en/class.pdo.php}} má svo nota til þess að mynda tenginguna sjálfa. 

\begin{example}[h]
\caption{Tenging við gagnagrunn með PDO}
\label{sql:k8d1}
\centering
\inputminted[frame=lines, fontfamily=courier]{php}{sql/k8d1.php}
\end{example}

Dæmi um hvernig öll skriptan gæti litið út má sjá á sýnidæmi \ref{sql:k8d1}.

Skoðum það sýnidæmi nánar.


\subsection{Að sækja gögn með PDO}
\subsection{Að birta gögn á vefsíðu}