\section{Views}
\section{Yfirlit yfir helstu gagnagrunnskerfi}
\label{undirkafli:helstu-gagnagrunnskerfi}
Gríðarmörg SQL-gagnagrunnskerfi eru til. Í þessum undirkafla verða þau gagnagrunnskerfi sem höfundur telur líklegast að nemendur Tækniskólans muni rekast á í náinni framtíð kynnt.

Grundvallaratriði SQL, sem farið er yfir í þessari bók, hafa miðast við notkun MySQL (\ref{undirkafli:mysql}). Hugmyndirnar á bak við þessi grundvallaratriði er sú sama í öllum gagnagrunnskerfum sem byggja á SQL, en \emph{útfærslan} er ekki endilega sú sama. Við skulum ekki búast við því að sýnidæmi bókarinnar keyri óbreytt í öllum gagnagrunnskerfum.
\subsection{MySQL}
\label{undirkafli:mysql}
\begin{marginfigure}
\caption{MySQL}
\label{mynd:mysql}
\centering
\includegraphics[width=\linewidth]{myndir/mysql}
\end{marginfigure}
MySQL\footnote{\url{http://www.mysql.com/}} er gagnagrunnskerfi í mikilli notkun, sérstaklega við vefsíðugerð.

Þökk sé útbreiðslunni er tiltölulega auðvelt að finna leiðbeiningar um notkun MySQL og uppsetningu MySQL-servera. Hægt er að vinna með MySQL í gegnum fjölmörg forritunarmál. MySQL er opið\footnote{e. \emph{open source}} og ókeypis.

MySQL hefur skilist að í nokkra hluta síðan það var fyrst gefið út. Upprunalegir höfundar kerfisins vinna nú við afbrigði sem heitir MariaDB\footnote{\url{https://mariadb.org/}}.
\subsection{PostgreSQL}
\begin{marginfigure}
\caption{PostgreSQL}
\label{mynd:postgresql}
\centering
\includegraphics[width=0.8\linewidth]{myndir/postgresql}
\end{marginfigure}
PostgreSQL\footnote{\url{http://www.postgresql.org/}} er gagnagrunnskerfi sem leggur mikla áherslu á að fylgja stöðlum og bjóða upp á ``rétta'' gagnavinnslu.

PostgreSQL er vinsælt meðal gagnagrunnssérfræðinga, m.a. vegna þess hversu mikla stjórn gagnagrunnstjórinn hefur yfir virkni kerfisins. Kennt er á PostgreSQL gagnasafnsfræðiáföngum Tækniskólans.

Fyrir utan það að styðja ``venjulegar'' SQL skipanir, þá býður PostgreSQL upp á forritunarmál, kallað PL/pgSQL (Procedural Language/PostgreSQL), til að auðvelda ýmsar aðgerðir. PL/pgSQL býður meðal annars upp á lykkjur og önnur tól sem kunnugleg eru úr forritunarmálum á borð við C\#.
\subsection{SQLite}
\begin{marginfigure}
\caption{SQLite}
\label{mynd:sqlite}
\centering
\includegraphics[width=\linewidth]{myndir/sqlite}
\end{marginfigure}
SQLite\footnote{\url{http://sqlite.org/}} er ólíkt flestum gagnagrunnskerfum að því leyti að ekki þarf að setja upp eiginlegt kerfi á tölvunni sem á að hýsa gagnagrunninn, allt forritið er ein skrá.

Smæðarinnar vegna vantar SQLite ýmsa eiginleika sem stærri gagnagrunnskerfi bjóða upp á, en það hentar sérstaklega vel til að nota sem hluta af stærri kerfum. SQLite má finna ``undir húddinu'' á mörgum forritum sem þurfa að geyma gögn.
\subsection{Microsoft SQL Server}
\begin{marginfigure}
\caption{SQL Server}
\label{mynd:sql-server}
\centering
\includegraphics[width=\linewidth]{myndir/sql-server}
\end{marginfigure}
SQL Server\footnote{\url{http://www.microsoft.com/sqlserver}} er gagnagrunnskerfi gefið út af Microsoft. Það er sniðið til að passa vel til keyrslu á Windows-vélum. SQL Server er helsti keppinautur Oracle gagnagrunnskerfisins þegar kemur að stórum gagnagrunnum.

Sú útgáfa af SQL sem notuð er til samskipta við SQL Server heitir Transact-SQL. T-SQL styður lykkjur og breytur.
\subsection{Oracle Database}
\label{undirkafli:oracle}
\begin{marginfigure}
\caption{Oracle Database}
\label{mynd:oracle}
\centering
\includegraphics[width=\linewidth]{myndir/oracle}
\end{marginfigure}
Gagnagrunnskerfi Oracle\footnote{\url{http://www.oracle.com/us/products/database/overview/}} er mest notaða gagnagrunnskerfi í heimi í dag. Það hefur verið í þróun áratugum saman og knýr marga af heimsins stærstu gagnagrunnum.

Útvíkkun Oracle á SQL til að styðja lykkjur og önnur algeng forritunaratriði er kallað PL/SQL (Procedural Language/Structured Query Language).
\section{Venslalíkanið}
\section{Að tengjast gagnagrunni með PHP}
\label{undirkafli:php}
Við höfum eytt miklum tíma í að skoða gagnagrunna sem sjálfstætt fyrirbrigði.

Í þessum undirkafla skoðum við loksins hvernig tengja má MySQL-gagnagrunn við PHP-forritskóða.
Slíkar tengingar eru teknar fyrir vegna þess hve algengar\footnote{Linux, Apache, PHP og MySQL mynda saman ``pakka'' sem oft er notaður sem ein heild. Pakkinn er nefndur eftir skammstöfun sinni, LAMP. Hann er í gríðarmikilli notkun.} þær eru í vefforritun.

Útskýringarnar gera ráð fyrir skilningi á ýmsum atriðum:
\begin{itemize}
 \item Hugtökum í vefsíðum - HTML, Javascript
 \item Keyrslu PHP-skripta
 \item Grundvallarmálfræði PHP
 \item IP-tölum
 \item Föllum, lykkjum og fylkjum\footnote{e. \emph{functions}, \emph{loops} og \emph{arrays}}
\end{itemize}
\subsection{Hlutverk gagnagrunna í vefsíðum}
Vefsíða sem byggð er upp á hefðbundinn hátt skiptist gróflega í tvo hluta - client og server. 

\begin{figure}
\centering
\caption{Hefðbundin uppbygging vefsíðu}
\label{mynd:uppbyggingvefsidu}
\color{red} Hingað kemur falleg mynd af client-server strúktúr vefsíðu.
\end{figure}

Client-hlutinn er sá hluti sem keyrður er á tölvu notandans. Í client-hluta eru HTML-tög túlkuð og Javascript-kóði keyrður, m.a.. Venjulega fer þessi vinna fram í vafra\footnote{e. \emph{browser}, t.d. Google Chrome, Firefox og Internet Explorer} notandans.

Server-hlutinn er margskiptur. Viðfangsefni okkar, PHP og MySQL, tilheyra þessum hluta. Oft keyra PHP og MySQL á sömu tölvu, sem er þá einfaldlega nefnd ``serverinn''.

Hlutverk gagnagrunnsins í þessari uppbyggingu er, eðli hans samkvæmt, það að halda utan um upplýsingar. PHP-hluti serversins sér um að eiga samskipti við gagnagrunninn og miðla upplýsingunum áfram til clientsins. Notandinn og tölva hans eiga aldrei bein samskipti við MySQL-serverinn.

\subsection{Uppsetning tengingar með PDO}
Gerum ráð fyrir að við séum með PHP-skriptu sem keyrir á vefþjóni. Til að tengjast MySQL-gagnagrunni þarf hún eftirfarandi upplýsingar:
\begin{itemize}
 \item Tengingarupplýsingar: Gagnagrunnsgerðina (hjá okkur alltaf MySQL), nafn gagnagrunnsins og staðsetningu hans
 \item Notandanafn MySQL-þjónsins
 \item Lykilorð notandans.
\end{itemize}
Klasann \href{http://www.php.net/manual/en/class.pdo.php}{PDO}\footnote{\url{http://www.php.net/manual/en/class.pdo.php}} má svo nota til þess að mynda tenginguna sjálfa.\footnote{Fleiri leiðir eru færar til að mynda tengingar af þessum toga, til dæmis \emph{mysqli} viðbótin. Þá PHP-viðbót sem einfaldlega heitir \emph{mysql} skal þó alls ekki nota - hún er úreld.}

\begin{example}[h]
\caption{Tenging við gagnagrunn með PDO}
\label{sql:k8d1}
\centering
\inputminted[frame=lines, fontfamily=courier]{php}{php/k8d1.php}
\end{example}

Dæmi um hvernig öll skriptan gæti litið út má sjá á sýnidæmi \ref{sql:k8d1}. Skoðum það sýnidæmi nánar. 

Fyrsta breytan sem er skilgreind í skriptunni (\verb|$source|) inniheldur tengingarupplýsingarnar. Þar kemur fyrst fram orðið \verb|mysql|, aðskilið frá gagnagrunnsnafninu og staðsetningu gagnagrunnins með tvípunkti. Sé ætlunin að skrifa skriptu af þessum toga til nota í eigin forriti þyrfti að skipta út \verb|testdb| fyrir nafn gagnagrunnsins sem nota skal. Sé MySQL-serverinn ekki að keyra á sömu tölvu og PHP-skriptan þarf einnig að breyta IP-tölunni í IP-tölu tölvunnar sem MySQL-serverinn keyrir á.\footnote{Server Upplýsingatækniskólans keyrir á IP-tölunni \emph{82.148.66.15}. Einnig má skrifa nafn hans, \emph{tsuts.tskoli.is}, í stað IP-tölunnar.}

Næstu tvær breytur innihalda notandanafn og lykilorð fyrir MySQL-serverinn. Þetta er sama notandanafn og lykilorð og notað var til að tengjast gagnagrunninum í kafla \ref{kafli:fyrstuskrefin}.

Þegar upplýsingarnar hafa verið geymdar er loks ``reynt'' að búa til tengingu með PDO. Takist það eru tengingarupplýsingarnar geymdar í  ``database handle'' breytu, nefnd \verb|$dbh| í sýnidæminu. Þessa breytu má síðan nýta til að eiga samskipti við gagnagrunninn.

Takist ekki að koma tengingunni á prentar skriptan út strenginn ``Tenging mistókst'' ásamt villuskilaboðunum sem berast.
\subsection{Að sækja gögn með PDO}
\subsection{Að birta gögn á vefsíðu}